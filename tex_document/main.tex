%%% ArXiv template from https://www.overleaf.com/latex/templates/an-arxiv-template/gbzmznbxvwpr

\documentclass[10pt]{article}
\usepackage{graphicx}
\baselineskip=16pt

\usepackage{indentfirst,csquotes}

\topmargin= .5cm
\textheight= 20cm
\textwidth= 32cc
\baselineskip=16pt

\evensidemargin= .9cm
\oddsidemargin= .9cm

\usepackage{amssymb,amsthm,amsmath}
\usepackage{xcolor,paralist,hyperref,titlesec,fancyhdr,etoolbox}


\usepackage[square,sort,comma,numbers]{natbib}
\newtheorem{theorem}{Theorem}[]
\newtheorem{definition}[theorem]{Definition}
\newtheorem{example}[theorem]{Example}
\newtheorem{lemma}[theorem]{Lemma}
\newtheorem{proposition}[theorem]{Proposition}
\newtheorem{corollary}[theorem]{Corollary}
\newtheorem{conjecture}[theorem]{Conjecture}


\hypersetup{ colorlinks=true, linkcolor=black, filecolor=black, urlcolor=black }
\def\proof{\noindent {\it Proof. $\, $}}
\def\endproof{\hfill $\Box$ \vskip 5 pt }


\usepackage{lipsum}

\title{Ongoing work on MCWAL} %%%%%%%%%%%%
\author{Pierre Borie}
\date{\today}




%%% PERSONAL ADD-ONS
\usepackage[square,sort,comma,numbers]{natbib}
\include{macros}

\begin{document}
	
	\maketitle
	
	\let\thefootnote\relax
	\footnotetext{MSC2020: Primary 00A05, Secondary 00A66.} %%%%%%%%%%
	
	\begin{abstract}
		\noindent This informal document  reflects the ongoing work and thinking on a algorithm for constrained nonlinear least squares. The current algorithm (rapper) name is MCWAL for Moindres Carr\'es With Augmented Lagrangian.
	\end{abstract} %%%%%%%%% 
	
	

	\section{Introduction}\label{sec:intro}
	
	We consider least squares problems subject to both nonlinear and linear constraints of the form
	\begin{equation}
		\label{eq:model_cnls}
		\begin{aligned}
			\min_{x\in \RR^n} \quad & \dfrac{1}{2} \|r(x)\|^2 \\
			\text{s.t.} \quad & c(x) = 0 \\
			& Ax = b \\
			& \ell \le x \le u,
		\end{aligned}
	\end{equation}
	where $r\colon \RR^n \to \RR^d$  and $c\colon \RR^n \to \RR^t$ are assumed to be nonlinear, potentially non convex, continuously differentiable functions, $\|\cdot\| $ denotes the euclidean norm, $A$ is a $m\times n \ ( m \le n)$, $b \in \RR^m$ and $\ell$ and $u$ are vectors in $\RR^n$. For the latter and without loss of generality, components can be set to $\pm \infty$ for unbounded parameters. In the context of least squares problems, components $r_i$ of the function $r$ are often denoted as the residuals.
	
	We will also refer to the linear constraints using the set notation
	\begin{equation}
		\label{eq:linear_constraints}
		X = \left\{ x \in \RR^n \ | \ Ax=b,\ \ell \le x \le u\right\}.
	\end{equation}
	
	
	
	\bibliographystyle{plainnat}
	\bibliography{refs}
\end{document}